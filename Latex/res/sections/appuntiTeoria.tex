\section{Appunti di teoria}
Questa sezione raccoglie alcuni appunti per un ripasso rapido.

\subsection{Tipi di collegamento}
Ci sono due tipi di collegamento:
\begin{itemize}
\item \textbf{Poin-to-Point} - la comunicazione avviene tra due macchine tramite dei pacchetti.
\item \textbf{Broadcast} - esiste un canale condiviso da tutte le macchine; una volta inviato un pacchetto viene ricevuto da tutti e viene processato solo dalla macchina che ha l'indirizzo indicato nel pacchetto; è possibile anche mandare un pacchetto a tutte le macchine della rete con un codice speciale nel campo indirizzo del pacchetto.
\end{itemize}

\subsection{Classificazione delle reti}
Classificazione delle reti in base alla distanza dei processori:
\begin{itemize}
\item PAN - Personal Area Network
\item LAN - Local (Stanza/Edificio)
\item MAN - Metropolitan (Città)
\item WAN - Wide (Nazione/Continente)
\item Internet (Pianeta)
\end{itemize}

\subsection{Architettura delle reti}
L'architettura delle reti è composta da livelli e protocolli.
Una rete è organizzata come una pila di livelli i quali offrono servizi ai livelli superiori, senza dare dettagli implementativi.

Un livello di un computer comunica con lo stesso livello di un altro computer.
Le entità su uno stesso livello sono detti peer (pari).
Tuttavia i dati non passano direttamente da un livello $n_A$ ad un livello $n_B$, ma dovranno essere passati ai sottolivelli di A per poi risalire i livelli di B.

Le regole e le convenzioni per la comunicazione tra livelli pari sono note come \textbf{protocolli} di livello. Un protocollo indica come deve avvenire la comunicazione tra le parti (es: il formato dei pacchetti).

Infine, tra i livelli sono presenti le \textbf{interfacce}, che definiscono le operazioni (o servizi) che il livello inferiore mette a disposizione del soprastante.

In una rete strutturata bene è possibile sostituire l'implementazione di un livello perchè quella nuova dovrà solo rendere disponibile al livello soprastante gli stessi servizi della vecchia.

\subsubsection{Progettare una rete}

I punti chiave da tenere in considerazione quando si progetta una rete sono:
\begin{itemize}
\item Affidabilità
\begin{itemize}
\item individuare gli errori e, se possibile, corregerli;
\item trovare un percorso valido tra sorgente e destinatario (routing/instradamento);
\end{itemize}
\item Evoluzione della rete
\item Allocazione delle risorse
\begin{itemize}
\item gestire una banda condivisa;
\item impedire che una sorgente veloce inondi un ricevitore lento (flow control) per evitare una congestione;
\end{itemize}
\item Sicurezza della rete
\end{itemize}

\subsubsection{Tipi di servizi offerti da un livello}
Un livello può offrire un servizio orientato alla connessione o senza connesione.

Il primo richiede di stabilire una connessione, usarla e rilasciarla una volta terminato. 
In questo servizio le informazioni inviate mantengono l'ordine di partenza.

Nel servizio senza connessione ogni pacchetto viene mandato al destinatario indipendentemente dai messaggi successivi, quindi non c'è sicurezza che l'ordine sia mantenuto. 

Inoltre c'è una questione legata all'affidabilità. 
Un servizio è considerato affidabile se riceve sempre tutti i pacchetti e di solito avviene tramite una conferma di ricezione. 
La conferma tuttavia rallenta le prestazioni del servizio, quindi è necessario valutare nei singoli casi se ne vale la pena: 
ad esempio in un servizio di streaming è più importante la velocità per un servizio più fluido piuttosto che l'affidabilità per una qualità migliore dell'audio/video.

\subsection{Modelli di reti}
Esistono due architetture di rete principali utilizzate come modello: OSI e TCP/IP. \\
Il primo utilizza 7 livelli:
\begin{enumerate}
\item fisico
\item data link
\item rete
\item trasporto
\item sessione
\item presentazione
\item applicazione
\end{enumerate}
Il secondo utilizza 4 livelli:
\begin{enumerate}
\item link
\item internet
\item trasporto
\item applicazione
\end{enumerate}
Il libro e il corso utilizza un modello ibrido a 5 livelli:
\begin{enumerate}
\item fisico
\item link
\item rete
\item trasporto
\item applicazione
\end{enumerate}


