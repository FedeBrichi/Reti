Questa sezione raccoglie alcuni appunti per un ripasso rapido.

\section{Capitolo 1 - Introduzione}

\subsection{Tipi di collegamento}
Ci sono due tipi di collegamento:
\begin{itemize}
\item \textbf{Poin-to-Point} - la comunicazione avviene tra due macchine tramite dei pacchetti.
\item \textbf{Broadcast} - esiste un canale condiviso da tutte le macchine;
una volta inviato un pacchetto viene ricevuto da tutti e viene processato solo dalla macchina che ha l'indirizzo indicato nel pacchetto;
è possibile anche mandare un pacchetto a tutte le macchine della rete con un codice speciale nel campo indirizzo del pacchetto.
\end{itemize}

\subsection{Classificazione delle reti}
Classificazione delle reti in base alla distanza dei processori:
\begin{itemize}
\item PAN - Personal Area Network
\item LAN - Local (Stanza/Edificio)
\item MAN - Metropolitan (Città)
\item WAN - Wide (Nazione/Continente)
\item Internet (Pianeta)
\end{itemize}

\subsection{Architettura delle reti}
L'architettura delle reti è composta da livelli e protocolli.
Una rete è organizzata come una pila di livelli i quali offrono servizi ai livelli superiori, senza dare dettagli implementativi.

Un livello di un computer comunica con lo stesso livello di un altro computer.
Le entità su uno stesso livello sono detti peer (pari).
Tuttavia i dati non passano direttamente da un livello $n_A$ ad un livello $n_B$, ma dovranno essere passati ai sottolivelli di A per poi risalire i livelli di B.

Le regole e le convenzioni per la comunicazione tra livelli pari sono note come \textbf{protocolli} di livello.
Un protocollo indica come deve avvenire la comunicazione tra le parti (es: il formato e significato dei pacchetti).

Infine, tra i livelli sono presenti le \textbf{interfacce}, che definiscono i \textit{servizi} (cioè un insieme di primitive/operazioni) che il livello inferiore mette a disposizione del soprastante.

In una rete strutturata bene è possibile sostituire l'implementazione di un livello perché quella nuova dovrà solo rendere disponibile al livello soprastante gli stessi servizi della vecchia.

\subsubsection{Progettare una rete}

I punti chiave da tenere in considerazione quando si progetta una rete sono:
\begin{itemize}
\item Affidabilità
\begin{itemize}
\item individuare gli errori e, se possibile, correggerli;
\item trovare un percorso valido tra sorgente e destinatario (routing/instradamento);
\end{itemize}
\item Evoluzione della rete
\item Allocazione delle risorse
\begin{itemize}
\item gestire una banda condivisa;
\item impedire che una sorgente veloce inondi un ricevitore lento (flow control) per evitare una congestione;
\end{itemize}
\item Sicurezza della rete
\end{itemize}

\subsubsection{Tipi di servizi offerti da un livello}
Un livello può offrire un servizio orientato alla connessione o senza connessione.

Il primo richiede di stabilire una connessione, usarla e rilasciarla una volta terminato. 
In questo servizio le informazioni inviate mantengono l'ordine di partenza.

Nel servizio senza connessione ogni pacchetto viene mandato al destinatario indipendentemente dai messaggi successivi, quindi non c'è sicurezza che l'ordine sia mantenuto. 

Inoltre c'è una questione legata all'affidabilità. 
Un servizio è considerato affidabile se riceve sempre tutti i pacchetti e di solito avviene tramite una conferma di ricezione. 
La conferma tuttavia rallenta le prestazioni del servizio, quindi è necessario valutare nei singoli casi se ne vale la pena: 
ad esempio in un servizio di streaming è più importante la velocità per un servizio più fluido piuttosto che l'affidabilità per una qualità migliore dell'audio/video.

Un esempio di primitive di un servizio orientato alla connessione sono: Listen, Connect, Accept, Receive, Send, Disconnect.

\subsection{Modelli di reti}
Esistono due architetture di rete principali utilizzate come modello: OSI e TCP/IP. \\
Il primo utilizza 7 livelli:
\begin{enumerate}
\item fisico
\item data link
\item rete
\item trasporto
\item sessione
\item presentazione
\item applicazione
\end{enumerate}
Il secondo utilizza 4 livelli:
\begin{enumerate}
\item link - livello  di accesso alla rete per spedire i pacchetti IP.
\item internet - livello che gestisce l'indirizzamento dei nodi e l'instradamento, assegnando ad ogni nodo un indirizzo IP e indicando il percorso migliore verso il destinatario.
\item trasporto - livello che gestisce la comunicazione, tramite protocollo TCP o UDP.
\item applicazione - livello più vicino all'utente che gestisce le sessioni e la presentazione; alcuni protocolli di questo livello sono HTTP e DNS.
\end{enumerate}
Il libro e il corso utilizza un modello ibrido a 5 livelli:
\begin{enumerate}
\item fisico - indica come vengono trasmessi i bits tramite segnali elettrici o simili.
\item link - indica come mandare i messaggi ai computer (es. Ethernet, 802.11).
\item rete - indica come combinare link multipli per spedire pacchetti tra computer distanti
\item trasporto - gestisce la comunicazione, tramite protocollo TCP o UDP.
\item applicazione - programmi che usano il network
\end{enumerate}

\newpage

\section{Capitolo 2 - Strato fisico}

L'obiettivo dello strato fisico è quello di trasportare i bits da una macchina ad un'altra.
Per farlo si possono usare diversi mezzi fisici con proprie caratteristiche.
I mezzi di trasmissione possono essere guidati (es: cavi) o non guidati (wireless, satelliti).

Le informazioni possono essere trasmesse su cavi sfruttando proprietà come voltaggio o corrente. 
Usando questi valori, si può modellare il comportamento del segnale e analizzarlo matematicamente.

\subsection{Serie di Fourier e Banda passante}

Fourier afferma che una funzione periodica con periodo T può essere costruita come la somma di un numero n di seni e coseni.
Da ciò deriva la serie di Fourier, una formula che permette di ricostruire la suddetta funzione periodica. 
Nel contesto delle reti, un segnale che ha durata finita può essere immaginato come un pattern che viene ripetuto con l'intervallo T e 2T identico all’intervallo 0 a T.
In questo modo, conoscendo periodo e ampiezza è possibile ricostruire la funzione del segnale, permettendo un analisi e modellazione più facile del segnale.
Il problema è che i mezzi di trasmissione perdono parte della potenza del segnale, generando una distorsione.
Un cavo riesce a trasmettere frequenze senza attenuazione in un intervallo che va da 0 a $f_c$ (frequenza di cutoff).
Questo intervallo è detto Banda passante (\textit{bandwidth}) e dipende da diversi fattori del mezzo di trasmissione (materiali, lunghezza e spessore di un cavo, ecc).
Le frequenze che vanno oltre vengono attenuate.
La frequenza di cutoff non è ben definita, quindi di solito si pone l'intervallo da 0 fino alla frequenza dove la potenza del segnale è dimezzata.
Questi sono detti segnali baseband.
A volte vengono utilizzati dei filtri che possono modificare la banda passante, per esempio alzando l'intervallo da un valore maggiore di zero: è il caso delle trasmissioni wireless.
Questi sono segnali passband.

\subsection{Mezzi di trasmissione guidati}

\subsubsection{Mezzi magnetici}
I mezzi magnetici sono un comune mezzo di trasporto per dati (cd, dcd, hdd) che in alcuni casi può risultare più conveniente se considerato un rapporto dimensioneDati/tempoTrasferimento.
Per esempio può essere più comodo trasportare un camion di hard disk piuttosto che spedire lo stesso quantitativo di dati tramite la rete. 
Anche se le connessioni stanno diventando sempre più veloci, in alcuni casi i mezzi magnetici possono rimanere la miglior soluzione (sempre da considerare il contesto).

\subsubsection{Il doppino}
Il doppino è un cavo composto da due conduttori in rame isolati di 1mm, attorcigliati in modo elicoidale (simile al DNA).
Questa forma permette di eliminare i campi elettromagnetici che si verrebbero a formare se fossero paralleli.
Un segnale è tramesso come differenza di voltaggio tra i due cavi, in modo da evitare i disturbi da rumori esterni(?).
L'uso più comune è per il telefono e l'accesso ad internet con l'ADSL. 
Il doppino può estendersi per chilometri, ma dopo certe distanze è necessario un ripetitore altrimenti il segnale diventa troppo attenuato.
Il prezzo del doppino è ridotto e ha una velocità di trasmissione moderata.
Si possono usare per segnali sia analogici che digitali e la larghezza di banda dipende dallo spessore del cavo e dalla distanza percorsa, tuttavia è limitata.

Esistono diversi tipi di cavi che utilizzano il doppino, come Cat 3 e Cat 5.
Questi consistono in due cavi isolati attorcigliati tra loro, raggruppati con altre coppie (4 totali), ricoperti da una protezione in plastica.
La differenza tra le due tipologie sta nel numero di spire per metro: maggior numero di spire significa una maggior qualità del segnale su lunghe distanze.
Esistono categorie superiori come Cat 6 e 7 che supportano segnali con una maggiore larghezza di banda (fino a 500 MHz).

\subsubsection{Il cavo coassiale}
Il cavo coassiale è un mezzo trasmissivo che permette una maggior larghezza di banda (fino a qualche GHz) rispetto al doppino grazie alla migliore schermatura,
permettendo di viaggiare a maggiori velocità a lunghe distanze.
In particolare, il cavo è composto da un nucleo conduttore in rame, ricoperto da un materiale isolante,
il quale a sua volta è ricoperto da un conduttore cilindrico intrecciato (tipo una rete), protetto da una guaina in plastica.
Esistono due tipi di cavi coassiali, usati in base al tipo di segnale.
Il 50$\Omega$ viene usato per i segnali digitali, mentre il 75$\Omega$ per i segnali analogici e per la televisione.
Questo cavo veniva usato anche nel campo della telefonia, ma ormai sta venendo rimpiazzato dalla fibra ottica. Viene ancora usato per la tv via cavo e per le MAN.

\subsubsection{Fibra ottica}
Un sistema di trasmissione ottico si basa su una fonte luminosa, un mezzo di trasmissione e un ricevitore.
La presenza di luce indica un 1 mentre l'assenza uno 0.
Nel caso della fibra ottica, si utilizza una sottilissima fibra di vetro come nucleo, attraverso la quale viaggia la luce. 
Alle estremità un ricevitore che "legge" i segnali luminosi e li traduce in segnali elettrici.
La buona riuscita della trasmissione del raggio luminoso sta negli indici di rifrazione dei componenti della fibra ottica. 
Grazie ad essi il raggio luminoso rimane nella fibra e continua il suo percorso fino a destinazione.
Infatti il core è ricoperto da un rivestimento di vetro (cladding) che ha un indice di rifrazione più basso, e a sua volta è ricoperto da uno strato protettivo in plastica.
Di solito le fibre sono raggruppati in fasci, a loro volta protetti da una guaina esterna.

In base allo spessore del nucleo, la fibra cambia il proprio nome.
Se un raggio al suo interno è propagato grazie ai rimbalzi della rifrazione, si dice multimodale (50 microns).
Se la fibra è abbastanza sottile da far procedere il raggio quasi in linea retta, si dice monomodale (8-10 microns).
Quest'ultima è più costosa ma più efficiente sulle lunghe distanze.

Ci sono tre modi per connettere la fibra ottica:
\begin{itemize}
\item collegamento della parte finale ad un connettore in apposite prese, con una perdita del 10-20\% del segnale luminoso ma una facile riconfigurazione del sistema
\item attaccate meccanicamente, cercando di allinearle al meglio, con una perdita del 10\% del segnale
\item fusione delle due parti, generando una piccola attenuazione
\end{itemize}

Le fonti luminose possono essere LED (basso data rate, multimodale, low cost) o laser semiconduttori (alto data rate, sia mono che multimodale, costoso).

Il ricevitore che converte il segnale luminoso in elettrico ha un limite di data rate di 100 Gbps.
Inoltre l'interferenza termica può risultare un problema, quindi conviene utilizzare raggi luminosi abbastanza potenti da essere rilevati.

La fibra ottica è una tecnologia relativamente recente, di conseguenza non tutti gli addetti hanno le conoscenze necessarie per installarla od utilizzarla correttamente;
può anche danneggiarsi se si piega troppo. 
Inoltre la trasmissione è monodirezionale, quindi sono richiesti due cavi per andata e ritorno, e le interfacce sono più costose.

Tuttavia ha una maggior ampiezza di banda, richiede meno ripetitori (uno ogni 50km contro uno ogni 5 di quelli in rame), il che porta ad un risparmio,
è più sottile, richiedendo quindi meno spazio, è più sicura perché non è possibile intercettare la luce ed infine è più adatta ai luoghi inospitali, in quanto subisce meno interferenze.

\subsection{Mezzi wireless}

\subsubsection{Spettro elettromagnetico}
Lo spostamento degli elettroni crea onde elettromagnetiche:
il numero di oscillazioni al secondo di un'onda è detta frequenza (misurata in Hz),
mentre la distanza tra due massimi è detta lunghezza d'onda (indicata da lambda).


\subsubsection{Trasmissioni radio}
\subsubsection{Trasmissioni a microonde}
\subsubsection{Infrarossi}
\subsubsection{Trasmissioni a donde luminose}

\subsection{Satelliti}

Un satellite è composto da tanti transponder che ascoltano una diversa porzione dello spettro elettromagnetico.
Quando riceve un segnale in arrivo, il relativo transponder lo amplifica e lo ritrasmette con frequenza diversa per evitare interferenze.
Esistono tre tipi di satelliti in base alla loro posizione: GEO, MEO, LEO.\\
Vai a §\ref{satelliti} per il confronto delle tre tipologie.

\subsection{Modulazione digitale e multiplexing}
La modulazione digitale è il processo di conversione tra bit e i segnali che li rappresentano. 

% ED.5 Inglese
% 2.5 DIGITAL MODULATION AND MULTIPLEXING, 125
% 2.5.1 Baseband Transmission, 125
% 2.5.2 Passband Transmission, 130
% 2.5.3 Frequency Division Multiplexing, 132
% 2.5.4 Time Division Multiplexing, 135
% 2.5.5 Code Division Multiplexing, 135
% 2.6 THE PUBLIC SWITCHED TELEPHONE NETWORK, 138
% 2.6.1 Structure of the Telephone System, 139
% 2.6.2 The Politics of Telephones, 142
% 2.6.3 The Local Loop: Modems, ADSL, and Fiber, 144
% 2.6.4 Trunks and Multiplexing, 152
% 2.6.5 Switching, 161

% ED.4 di riferimento
% 2.5. The Public Switched Telephone Network
% 2.5.1. Structure of the Telephone System
% 2.5.2. The Politics of Telephones
% 2.5.3. The Local Loop: Modems, ADSL, and Wireless
% Modems
% Digital Subscriber Lines
% Wireless Local Loops ----NO
% 2.5.4. Trunks and Multiplexing
% Frequency Division Multiplexing
% Wavelength Division Multiplexing
% Time Division Multiplexing
% SONET/SDH ----NO
% 2.5.5. Switching
% Circuit Switching
% Message Switching
% Packet Switching


\subsection{Sistema telefonico mobile}


\newpage
\section{Capitolo 3 - Strato data link}


\newpage
\section{Capitolo 4 - Sottostrato MAC}


\newpage
\section{Capitolo 5 - Strato rete}


\newpage
\section{Capitolo 6 - Strato trasporto}


\newpage
\section{Capitolo 7 - Strato applicazione} % SOLO DNS

Lo strato applicazione e dove si trovano effettivamente le applicazioni.
Ci sono comunque dei protocolli di supporto per permettere alle applicazioni di funzionare.
Uno di questi è il DNS.

\subsection{DNS}

Il DNS (Domain name system) è un protocollo per la gestione dei nomi.
I siti web e le altre risorse possono essere accedute direttamente tramite l'indirizzo IP,
ma non risulta molto user-friendly in quanto per l'utente è difficile memorizzare l'indirizzo della risorsa richiesta.
Il DNS serve per tradurre gli indirizzi IP in nomi comprensibili e viceversa.
Infatti, se per l'utente il nome è più comprensibile, il network comprende solo l'indirizzo IP, per cui è richiesta la traduzione inversa da nome a indirizzo.
Un altro problema che ha portato alla creazione del DNS era la necessità di un sistema centralizzato, che permettesse di gestire i nomi senza avere duplicati.

Il DNS quindi è un sistema centralizzato con un database distribuito dove viene implementato lo schema gerarchico dei nomi basato su dominio. 
Viene utilizzato principalmente per mappare indirizzi IP e nomi del relativo host.
Per mappare un nome con il suo indirizzo IP, viene chiamata una procedura detta \texttt{resolver} con il nome come parametro.
Questa invia la query al server DNS e viene ritornato l'indirizzo IP; sia la query, sia la risposta sono spediti come pacchetti UPD.

La gerarchia dei nomi è divisa in diversi livelli. 
Internet è diviso in più di 250 domini di primo livello contenenti i vari host. 
Ogni dominio è partizionato in sottodomini, anch'essi partizionati ecc.
Il primo livello è distinto tra generici (com, org, edu, ecc) e nazioni (it, uk, jp, ecc).
Il secondo livello spesso indica l'azienda.
Il nome di un dominio può essere assoluto o relativo.
Se termina con il punto è assoluto; se è relativo, il reale significato dipende dal contesto.
I nomi sono case-insensitive e di lunghezza massima di 255 caratteri.

Ogni dominio può essere associato ad un insieme di record delle risorse.
Quando il DNS riceve un nome, restituisce i record associati, tra cui quello che indica l'indirizzo IP.
Un record è identificato dalla quintupla:\\
\texttt{Domain_name Time_to_live Class Type Value}
Il Type A è il più importante in quanto è il record che contiene l'indirizzo IP. Questo record può non essere univoco.

Il server del DNS non è unico, altrimenti verrebbe sovraccaricato. Quindi i nomi sono divisi in zone, che risiedono su server diversi. 

\newpage
\section{Capitolo 8 - Sicurezza}















%%%%%%%%%%%%%%%%%%%%%%%%%%%%%%%%%%%%%%%%%%%%%%%%%%%%%%%%%

% Argomenti:

% Capitolo 1, tutto tranne le sezioni da 1.5 a 1.9 incluse. % OK %
% Capitolo 2, tutto tranne:

% la sottosezione Wireless Local Loops di 2.5.3
% la sottosezione SONET/SDH di 2.5.4
% la sezione 2.7 Cable Television

% Capitolo 3, tutto esclusi tutti i listati in C, e la sezione 3.5 Protocol Verification
% Capitolo 4, tutto tranne:

% la sezione 4.2.5 Wavelength Division Multiple Access Protocols
% la sezione 4.3.9 IEEE 802.2: Logical Link Control
% la sezione 4.4.4 The 802.11 Frame Structure
% la sezione 4.5 Broadband Wireless
% la sezione 4.6 Bluetooth
% la sottosezione 4.7.6 Virtual LANs

% Capitolo 5, tutto tranne:

% nella sezione 5.2.2, la descrizione dell'algoritmo di Dijkstra (incluso il listato)
% la sottosezione Computing the New Routes di 5.2.5
% dalla sezione 5.2.8 Multicast Routing alla sezione 5.2.11 Node Lookup in Peer-to-Peer Networks
% la sottosezione The Warning Bit di 5.3.4
% la parte di 5.4.2 da Resource Reservation in poi
% dalla sezione 5.4.3 Integrated Services alla sezione 5.4.5 Label Switching and MPLS
% la sezione 5.5.5 Tunneling
% la sezione 5.5.7 Fragmentation
% la descrizione di RARP e BOOTP nella sottosezione RARP, BOOTP, and DHCP
% la sezione 5.6.7 Mobile IP

% Capitolo 6, le seguenti parti:

% il three-way handshake nella sezione 6.2.2 Connection Establishment
% la sezione 6.2.3 Connection Release
% la sezione 6.4.1 Introduction to UDP
% dalla sezione 6.5.1 Introduction to TCP alla sezione 6.5.6 TCP Connection Release

% Capitolo 7, le seguenti parti:
% la sezione 7.1 DNS - The Domain Name System  % OK %

% Capitolo 8, le seguenti parti:

% la sezione 8.1 Cryptography. tranne le sottosezioni transposition ciphers (8.1.3) e Quantum Cryptography (in 8.1.4)
% la sezione 8.2 Symmetric-Key Algorithms, tranne i cipher mode di tipo Block Chaining, Feedback, e Counter nella sottosezione Cipher Modes (8.2.3)
% la sezione 8.3 Public-Key Algorithms
% la sottosezione 8.4.3 Message Digests
% la sottosezione 8.6.1 IPSec nella versione ESP (Encapsulating Security Payload)
% la sottosezione 8.6.2 Firewalls
% la sottosezione 8.6.4. Wireless Security, tranne la parte Bluetooth Security
% il replay attack (sottosezione 8.7.3)
% il DNS Spoofing (sottosezione 8.9.2), tranne la parte Self-Certifying Names