\section{Cos’è un cifrario a sostituzione? e a trasposizione?}

Le reti inizialmente venivano utilizzate da ricercatori universitari per scambiarsi e-mail, e dalle aziende per condividere le stampanti, di conseguenza la sicurezza aveva un ruolo marginale.
Oggi le reti sono utilizzate da milioni di persone per fare acquisti, lavorare con la banca o per documenti importanti. Questo fa diventare la sicurezza qualcosa di fondamentale e ricercato in quanto sempre più persone malintenzionate cercano di rubare dati sensibili.
La crittografia serve a rendere un messaggio non comprensibile/leggibile a persone non autorizzate a leggerlo.
Per cifrare un messaggio ci sono diverse tecniche, una di queste è il cifrario a sostituzione (per cifrario s’intende una trasformazione carattere per carattere, senza considerare la struttura linguistica del messaggio). Uno dei cifrari più antichi che si conoscono è il cifrario di Cesare.
Il sistema generale sta nel sostituire appunto un carattere / coppie di lettere/sillabe/ecc. con altre.
Un altro tipo di cifrario è il cifrario a trasposizione che riordinano le lettere senza mascherarle come fa il cifrario a sostituzione.
Un esempio è la trasposizione colonnare che funziona come segue: tramite una parola chiave si numerano le colonne, il testo in chiaro va disposto di seguito sulle colonne. Successivamente si ordinano le colonne in base alla parola chiave, ad ogni lettera viene dato un valore dipendente dal valore della lettera nell’alfabeto, successivamente si riscrive per colonne il testo cifrato.
Le differenze sostanziali tra i due è che nel cifrario a sostituzione l’ordine rimane invariato ma le lettere vengono mascherate, in quello a trasposizione le lettere non vengono mascherate e il testo viene mescolato (secondo opportuni criteri).
